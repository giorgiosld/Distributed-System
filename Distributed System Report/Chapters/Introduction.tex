\chapter{Introduction}

Distributed systems are networks of autonomous computers that communicate and coordinate their actions by passing messages. These systems aim to achieve a common goal by leveraging the combined computational power and resources of multiple independent machines. The main characteristics of distributed systems include concurrency, scalability, fault tolerance, resource sharing, and transparency. Examples of distributed systems include cloud computing platforms, peer-to-peer networks, and large-scale web applications. These systems are widely used due to their ability to handle large volumes of data and provide high availability and reliability.
Leader election is a fundamental problem in distributed systems. It involves selecting a node, known as the leader, to coordinate the actions of other nodes. The leader is responsible for tasks such as managing resources, coordinating communication, and making decisions that affect the entire system. Efficient leader election algorithms are crucial for maintaining system consistency and performance.

This report aims to provide a comprehensive comparative analysis of various leader election algorithms, focusing on their efficiency, performance, and scalability. The justification for this report lies in the critical role that leader election plays in maintaining system coherence, ensuring fault tolerance, and optimizing resource utilization. The central research question addresses which leader election algorithms offer the best trade-offs between complexity, communication overhead, robustness, and convergence time in different distributed environments.
Additionally, this study explores the evolving role of leader election in blockchain technology, highlighting its significance in ensuring decentralized consensus, enhancing security, and maintaining the integrity of distributed ledgers. This aspect is particularly important as blockchain represents a paradigm shift in distributed systems, introducing new challenges and opportunities for leader election mechanisms.
The practical applications of these concepts have been illustrated through Python implementations of the leader election algorithms, which are available for review and further development in a publicly accessible \href{https://github.com/giorgiosld/Distributed-System/tree/main}{GitHub repository}. \\
The report is organized as follows:

\begin{itemize}
    \item \textbf{Chapter 1: Background} - This chapter provides an overview of distributed systems and the importance of leader election algorithms. It also discusses various election algorithms used in different scenarios.
    \item \textbf{Chapter 2: Algorithms} - This chapter delves into the specifics of various leader election algorithms, including the Bully algorithm, Ring algorithm, Proof of Work (PoW), and Proof of Stake (PoS). Each algorithm is described in detail, along with its system model, pseudocode, and analysis.
    \item \textbf{Chapter 3: Comparative Analysis of Leader Election Algorithms} - This chapter compares the leader election algorithms based on several critical criteria, such as complexity, communication overhead, robustness, and convergence time.
    \item \textbf{Chapter 4: Conclusion} - The final chapter summarizes the findings of the comparative analysis and discusses the implications of the results-
\end{itemize}
Through detailed examination and comparison of key algorithms, this report aims to highlight significant improvements in efficiency and performance over time. These advancements reflect the evolution of leader election algorithms from traditional distributed systems to modern applications, addressing new challenges posed by decentralized and large-scale environments.
