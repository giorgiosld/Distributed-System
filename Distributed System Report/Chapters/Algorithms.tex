\chapter{Algorithms}

This section delves into the specifics of various leader election algorithms employed in distributed systems, namely the Bully algorithm, Ring algorithm, Proof of Work (PoW), and Proof of Stake (PoS), providing an overview of each algorithm followed by an in-depth explanation of their mechanisms and the formal mathematical theories that underpin them, highlighting the critical role of leader election algorithms in ensuring coordination, efficiency, and fault tolerance in distributed systems, and exploring how the choice of algorithm affects system performance in terms of communication overhead, convergence time, and robustness.

\section{Bully}

%The Bully algorithm is a well-known method for leader election in distributed systems. It is particularly effective in scenarios where nodes can directly communicate with each other. The algorithm proceeds as follows:
The Bully algorithm is a well-known method for leader election in distributed systems to dynamically select a coordinator from among a group of processes.  It is particularly effective in scenarios where nodes can directly communicate with each other and it operates under the assumption that each process has a unique identifier (ID), and the process with the highest ID among the non-faulty processes is elected as the coordinator. It is favored in environments where direct communication between all nodes is possible, making it suitable for smaller, tightly-coupled distributed systems. Its straightforward approach and deterministic nature ensure that a leader is always elected if the system remains functional, thus providing robust coordination \cite{Ref9}.

\subsection{System Model}
Consider a distributed system consisting of \(n\) processes \(P_1, P_2, \ldots, P_n\). Each process has a unique identifier (ID) such that:

\[
ID(P_i) > ID(P_j) \quad \text{for} \quad i > j
\]
The Bully Algorithm can be described in the following steps:

\subsubsection{Election Initialization}

When a process \(P_i\) detects that the leader (coordinator) is not functioning, it initiates an election by sending an election message to all processes with higher IDs.

\begin{equation}
\forall P_j \quad \text{where} \quad ID(P_j) > ID(P_i), \quad P_i \rightarrow P_j: \text{Election}
\end{equation}

\subsubsection{Response to Election Message}

Each process \(P_j\) that receives the election message responds with an OK message to \(P_i\) and starts its own election if it is not already doing so.

\begin{equation}
P_j \rightarrow P_i: \text{OK}
\end{equation}

\subsubsection{Election Process}

The process waits for a time interval \(T\). If no process with a higher ID responds within \(T\), it declares itself the leader.

\begin{equation}
\text{If no response within } T, \quad P_i \rightarrow \forall P_k: \text{Coordinator}(ID(P_i))
\end{equation}
If a process receives a coordinator message, it updates its leader to the process that sent the message.

\subsubsection{Algorithm Termination}

The algorithm terminates when a single leader is elected. Each process \(P_i\) has a variable \(Leader_i\) which stores the ID of the elected leader:

\begin{equation}
Leader_i = ID(P_k) \quad \text{where} \quad P_k \text{ is the elected leader}
\end{equation}

\subsection{Pseudocode}

The following pseudocode provides a clear and concise representation of the Bully Algorithm. The pseudocode is structured into several key parts, reflecting the steps each process follows to ensure a leader is elected within a distributed system.

\begin{itemize}
    \item \textbf{Election Initialization:} A process initiates an election upon detecting that the leader is no longer functioning by sending an election message to all processes with higher IDs.
    \item \textbf{Response to Election Message:} A process that receives an election message responds with an acknowledgment and may start its own election if it has a higher ID.
    \item \textbf{Election Process:} The initiating process waits for responses. If no response is received within a specified time interval, it declares itself the leader.
    \item \textbf{Coordinator Declaration:} The process that declares itself as the leader sends a coordinator message to all other processes to inform them of the new leader.
\end{itemize}
The pseudocode for the Bully Algorithm is as follows:

\begin{verbatim}
def initiate_election(P):
    higher_id_processes = get_higher_id_processes(P)
    if not higher_id_processes:
        send_victory_message(P)
    else:
        send_election_message(higher_id_processes)
        wait_for_responses()

def handle_election_message(sender):
    if sender.id < self.id:
        send_answer_message(sender)
        initiate_election(self)

def handle_victory_message(sender):
    self.coordinator = sender
\end{verbatim}

\subsubsection{Explanation of Pseudocode}

\begin{itemize}
    \item \texttt{initiate\_election(P)}: This function is called when process \(P\) detects that the current coordinator is not functioning. It identifies all processes with higher IDs and sends them election messages. If no higher ID processes exist, \(P\) declares itself the leader by sending a victory message.
    \item \texttt{handle\_election\_message(sender)}: This function is called when a process receives an election message from another process (sender). If the sender's ID is lower than the process's own ID, it sends an acknowledgment and initiates its own election process.
    \item \texttt{handle\_victory\_message(sender)}: This function is called when a process receives a victory message. It updates its coordinator variable to reflect the sender as the new coordinator.
\end{itemize}
This pseudocode provides a high-level view of the Bully Algorithm's operations, highlighting the interactions between processes and the steps involved in electing a new coordinator in a distributed system.

\subsection{Analysis}
The analysis of the Bully Algorithm focuses on several key metrics: correctness, liveness, safety, and complexity. These metrics are critical in evaluating the performance and reliability of the algorithm in distributed systems.

\subsubsection{Correctness}

The correctness of the Bully Algorithm is guaranteed by its design. The algorithm ensures that the process with the highest ID among the non-faulty processes is always elected as the leader. This is achieved by having each process with a higher ID preempt the election process initiated by processes with lower IDs. Consequently, no two processes can simultaneously declare themselves as leaders, thus maintaining the integrity of the election process.

\subsubsection{Liveness}

Liveness is a key property that ensures that the election process eventually completes, and a leader is elected. In the Bully Algorithm, every process either declares itself the leader if it receives no response from higher-ID processes or acknowledges another process as the leader. This guarantees that the system does not enter a state of indefinite waiting, and a leader is always elected, ensuring the continued operation of the distributed system.

\subsubsection{Safety}

Safety is maintained in the Bully Algorithm by ensuring that at most one leader is elected. The algorithm's design allows processes with higher IDs to preempt those with lower IDs, preventing multiple processes from simultaneously declaring themselves as leaders. This mechanism ensures that there is always a single, unique leader in the system, thus avoiding conflicts and ensuring consistent coordination.

\subsubsection{Complexity}

The complexity of the Bully Algorithm can be evaluated in terms of time and communication overhead:
\begin{itemize}
    \item \textbf{Time Complexity}: the worst-case time complexity of the Bully Algorithm is \(O(n^2)\), where \(n\) is the number of processes. This scenario occurs when each process \(P_i\) must send messages to all processes with higher IDs, and each of those processes must respond. The quadratic complexity can be a significant drawback in large-scale systems, where the number of messages and the time required for the election process can become substantial.
    \item \textbf{Communication Overhead}: given \(n\) processes, the message complexity in the worst case can be described as:
\[ O(n^2) \]
where:
\[ (n-1) + (n-2) + \cdots + 1 = \frac{n(n-1)}{2} \approx O(n^2) \]
This complexity arises because each process with a lower ID sends Election messages to all higher ID processes, and each higher ID process responds with an Answer message if it is active. Consequently, the communication overhead is high, especially in the worst-case scenario. Each process needs to communicate with all higher-ID processes, resulting in a large number of message exchanges. This can lead to network congestion and increased latency, particularly in systems with a large number of nodes.

\end{itemize}

\subsubsection{Scalability}

Due to its high communication overhead and quadratic time complexity, the Bully Algorithm is less suitable for large-scale distributed systems. It is more effective in smaller, tightly-coupled systems where direct communication between nodes is feasible, and the number of processes is manageable.

\subsubsection{Fault Tolerance}

The algorithm can handle the failure of any process, including the current leader, by initiating a new election process. However, in asynchronous systems, it may face challenges with incorrect leader election if timeouts are not accurately set. This can lead to temporary inconsistencies until a new leader is correctly elected.

%\subsection{Analysis}
%\subsubsection{Correctness}

%The correctness of the Bully Algorithm can be shown by proving that it guarantees the election of the highest ID process as the leader and that no two processes simultaneously declare themselves as leaders.

%\subsubsection{Liveness}

%Every process eventually either declares itself the leader or acknowledges another process as the leader.

%\subsubsection{Safety}

%At most one leader is elected since processes with higher IDs can preempt the election process initiated by processes with lower IDs.

%\subsubsection{Complexity}

%The time complexity of the Bully Algorithm in the worst case is \(O(n^2)\), where \(n\) is the number of processes. This occurs when each process \(P_i\) initiates an election and every other process \(P_j\) responds.

\section{Ring}

% The Ring algorithm is another widely-used method for leader election in distributed systems. It is particularly suitable for systems with a logical ring topology. The algorithm proceeds as follows:
The Ring algorithm is a popular method for leader election in distributed systems with a logical ring topology. This algorithm is effective in scenarios where nodes are organized in a ring and each node can communicate directly with its immediate neighbors. It operates under the assumption that each process has a unique identifier (ID), and the process with the highest ID among the non-faulty processes is elected as the coordinator. The algorithm is favored in environments where nodes are arranged in a circular structure, making it suitable for systems with a pre-defined logical ring topology. Its straightforward approach ensures that a leader is always elected if the system remains functional, thus providing robust coordination \cite{Ref10}.

\subsection{System Model}
Consider a distributed system consisting of \(n\) processes \(P_1, P_2, \ldots, P_n\) arranged in a logical ring. Each process has a unique identifier (ID) such that:

\[
ID(P_i) > ID(P_j) \quad \text{for} \quad i > j
\]
The Ring Algorithm can be described in the following steps:

\subsubsection{Election Initialization}

When a process \(P_i\) detects that the leader (coordinator) is not functioning, it initiates an election by sending an election message to its immediate neighbor in the ring.

\begin{equation}
P_i \rightarrow P_{i+1}: \text{Election}(ID(P_i))
\end{equation}

\subsubsection{Message Passing}

Each process that receives an election message compares the received ID with its own. If the received ID is higher, it forwards the message to its neighbor; otherwise, it replaces the ID with its own and then forwards it.

\begin{equation}
\text{If } ID(P_j) < ID(P_i), \quad P_j \rightarrow P_{j+1}: \text{Election}(ID(P_j))
\end{equation}

\subsubsection{Election Process}

The election message circulates around the ring until it returns to the initiator. The initiator then sends a coordinator message with the highest ID found.

\begin{equation}
P_i \rightarrow \forall P_k: \text{Coordinator}(ID(P_{\text{max}}))
\end{equation}

If a process receives a coordinator message, it updates its leader to the process that sent the message.

\subsubsection{Algorithm Termination}

The algorithm terminates when a single leader is elected. Each process \(P_i\) has a variable \(Leader_i\) which stores the ID of the elected leader:

\begin{equation}
Leader_i = ID(P_{\text{max}}) \quad \text{where} \quad P_{\text{max}} \text{ is the elected leader}
\end{equation}

\subsection{Pseudocode}

The following pseudocode provides a clear and concise representation of the Ring Algorithm. The pseudocode is structured into several key parts, reflecting the steps each process follows to ensure a leader is elected within a distributed system.

\begin{itemize}
    \item \textbf{Election Initialization:} A process initiates an election upon detecting that the leader is no longer functioning by sending an election message to its immediate neighbor.
    \item \textbf{Message Passing:} A process that receives an election message forwards it with the highest ID seen so far.
    \item \textbf{Election Process:} The election message circulates until it returns to the initiator, who then sends a coordinator message.
    \item \textbf{Coordinator Declaration:} The process that initiated the election sends a coordinator message to all other processes to inform them of the new leader.
\end{itemize}
The pseudocode for the Ring Algorithm is as follows:

\begin{verbatim}
def initiate_election(P):
    send_election_message(P.next, P.id)

def handle_election_message(sender, id):
    if id > self.id:
        forward_message(self.next, id)
    else:
        forward_message(self.next, self.id)

def handle_coordinator_message(id):
    self.coordinator = id
\end{verbatim}

\subsubsection{Explanation of Pseudocode}

\begin{itemize}
    \item \texttt{initiate\_election(P)}: This function is called when process \(P\) detects that the current coordinator is not functioning. It sends an election message to its immediate neighbor with its own ID.
    \item \texttt{handle\_election\_message(sender, id)}: This function is called when a process receives an election message. It forwards the message with the highest ID seen so far.
    \item \texttt{handle\_coordinator\_message(id)}: This function is called when a process receives a coordinator message. It updates its coordinator variable to reflect the new leader.
\end{itemize}
This pseudocode provides a high-level view of the Ring Algorithm's operations, highlighting the interactions between processes and the steps involved in electing a new coordinator in a distributed system.

\subsection{Analysis}
The analysis of the Ring Algorithm focuses on several key metrics: correctness, liveness, safety, and complexity. These metrics are critical in evaluating the performance and reliability of the algorithm in distributed systems.

\subsubsection{Correctness}

The correctness of the Ring Algorithm is guaranteed by its design. The algorithm ensures that the process with the highest ID among the non-faulty processes is elected as the leader. This is achieved by having each process compare the received ID with its own and forward the highest ID seen so far. Consequently, no two processes can simultaneously declare themselves as leaders, maintaining the integrity of the election process.

\subsubsection{Liveness}

Liveness is a key property that ensures that the election process eventually completes, and a leader is elected. In the Ring Algorithm, every process either declares itself the leader if it has the highest ID or forwards the highest ID seen so far. This guarantees that the system does not enter a state of indefinite waiting, and a leader is always elected, ensuring the continued operation of the distributed system.

\subsubsection{Safety}

Safety is maintained in the Ring Algorithm by ensuring that at most one leader is elected. The algorithm's design allows only the highest ID to be propagated around the ring, preventing multiple processes from simultaneously declaring themselves as leaders. This mechanism ensures that there is always a single, unique leader in the system, thus avoiding conflicts and ensuring consistent coordination.

\subsubsection{Complexity}

The complexity of the Ring Algorithm can be evaluated in terms of time and communication overhead:
\begin{itemize}
    \item \textbf{Time Complexity}: the time complexity of the Ring Algorithm in the worst case is \(O(n)\), where \(n\) is the number of processes. This scenario occurs when the election message has to circulate around the entire ring before a leader is elected. Each process forwards the election message to its immediate neighbor, resulting in linear time complexity.
    \item \textbf{Communication Overhead}: the communication overhead of the Ring Algorithm is moderate. Each process sends a single message to its neighbor, and the total number of messages exchanged is proportional to the number of processes, \(n\). Given \(N\) processes, the message complexity in the worst case can be described as:
\[
\Theta(N) = N
\]
This complexity arises because each process forwards the election message to its neighbor exactly once.
\end{itemize}

\subsubsection{Scalability}

Due to its linear communication overhead and time complexity, the Ring Algorithm is well-suited for distributed systems with a logical ring topology. It scales efficiently with the number of processes and maintains performance even in larger systems.

\subsubsection{Fault Tolerance}

The algorithm can handle the failure of any process, including the current leader, by initiating a new election process. Since the election message circulates around the ring, the algorithm is resilient to node failures as long as the ring remains intact. However, the algorithm's efficiency can be impacted if multiple failures occur simultaneously, necessitating multiple election cycles.

In summary, the Ring Algorithm provides a reliable and efficient method for leader election in distributed systems with a ring topology. Its linear time complexity and moderate communication overhead make it suitable for scalable systems, while its robustness ensures reliable leader election even in the presence of node failures.

%\subsection{Analysis}
%\subsubsection{Correctness}

%The correctness of the Ring Algorithm can be shown by proving that it guarantees the election of the highest ID process as the leader and that no two processes simultaneously declare themselves as leaders.

%\subsubsection{Liveness}

%Every process eventually either declares itself the leader or acknowledges another process as the leader.

%\subsubsection{Safety}

%At most one leader is elected since the election message ensures that only the highest ID is propagated.

%\subsubsection{Complexity}

%The time complexity of the Ring Algorithm in the worst case is \(O(n)\), where \(n\) is the number of processes. This occurs when the election message has to circulate around the entire ring.

\section{Proof of Work}

% The Proof of Work (PoW) algorithm is a widely-used method for leader election in blockchain-based distributed systems. It is particularly suitable for decentralized networks where security and consensus are critical. The algorithm proceeds as follows:
The Proof of Work (PoW) algorithm is a widely-used method for leader election in blockchain-based distributed systems. This algorithm is effective in scenarios where nodes compete to solve a cryptographic puzzle, with the first node to solve the puzzle being elected as the leader. It operates under the assumption that each process has a unique computational power, and the process that solves the puzzle first gets the right to add a new block to the blockchain. The algorithm is favored in environments where security, decentralization, and consensus are critical, making it suitable for blockchain networks. Its probabilistic approach ensures that a leader is always elected, thus providing robust coordination \cite{Ref11}.

\subsection{System Model}
Consider a distributed system consisting of \(n\) processes \(P_1, P_2, \ldots, P_n\) that are part of a blockchain network. Each process competes to solve a cryptographic puzzle, and the process that solves it first is elected as the leader:

\[
\text{Solve}(P_i) \quad \text{if} \quad \text{Hash}(P_i) < \text{Target}
\]
The Proof of Work Algorithm can be described in the following steps:

\subsubsection{Puzzle Initialization}

Each process \(P_i\) attempts to solve a cryptographic puzzle by finding a nonce that, when hashed with the block data, produces a hash value less than a predefined target.

\begin{equation}
\text{Hash}(P_i \oplus \text{nonce}) < \text{Target}
\end{equation}

\subsubsection{Competition Process}

Each process continuously attempts to find a valid nonce. The first process to find a valid nonce broadcasts its solution to the network.

\begin{equation}
P_i \rightarrow \forall P_j: \text{Solution}(\text{nonce})
\end{equation}

\subsubsection{Validation and Consensus}

Other processes verify the solution. If the solution is valid, the process that found the nonce is elected as the leader and gets the right to add a new block to the blockchain.

\begin{equation}
\text{If valid, } P_j \rightarrow \forall P_k: \text{New Block}(P_i)
\end{equation}

\subsubsection{Algorithm Termination}

The algorithm terminates when a single leader is elected and a new block is added to the blockchain. Each process \(P_i\) updates its blockchain to include the new block:

\begin{equation}
\text{Blockchain}_i = \text{Blockchain}_i + \text{New Block}(P_i)
\end{equation}

\subsection{Pseudocode}

The following pseudocode provides a clear and concise representation of the Proof of Work Algorithm. The pseudocode is structured into several key parts, reflecting the steps each process follows to ensure a leader is elected within a blockchain network.

\begin{itemize}
    \item \textbf{Puzzle Initialization:} Each process attempts to solve a cryptographic puzzle by finding a valid nonce.
    \item \textbf{Competition Process:} Each process continuously attempts to find a valid nonce and broadcasts its solution if successful.
    \item \textbf{Validation and Consensus:} Other processes verify the solution and update the blockchain with the new block.
\end{itemize}
The pseudocode for the Proof of Work Algorithm is as follows:

\begin{verbatim}
def find_nonce(P):
    while True:
        nonce = generate_nonce()
        if hash(P.data + nonce) < Target:
            broadcast_solution(P, nonce)
            break

def handle_solution(sender, nonce):
    if validate_solution(sender.data, nonce):
        update_blockchain(sender, nonce)
\end{verbatim}

\subsubsection{Explanation of Pseudocode}

\begin{itemize}
    \item \texttt{find\_nonce(P)}: This function is called by each process to find a valid nonce. It generates nonces until a valid one is found and then broadcasts the solution.
    \item \texttt{handle\_solution(sender, nonce)}: This function is called when a process receives a solution. It validates the solution and updates the blockchain if the solution is correct.
\end{itemize}
This pseudocode provides a high-level view of the Proof of Work Algorithm's operations, highlighting the interactions between processes and the steps involved in electing a new leader and adding a new block to the blockchain.

\subsection{Analysis}
The analysis of the Proof of Work (PoW) Algorithm focuses on several key metrics: correctness, liveness, safety, and complexity. These metrics are critical in evaluating the performance and reliability of the algorithm in blockchain-based distributed systems.

\subsubsection{Correctness}

The correctness of the Proof of Work Algorithm is guaranteed by its design. The algorithm ensures that the first process to solve the cryptographic puzzle is elected as the leader. This is achieved by having each process attempt to find a nonce that, when hashed with the block data, produces a hash value below a predefined target. Consequently, only one process can solve the puzzle first, thus maintaining the integrity of the election process.

\subsubsection{Liveness}

Liveness is a key property that ensures that the election process eventually completes, and a leader is elected. In the Proof of Work Algorithm, every process continuously attempts to find a valid nonce. Eventually, one process will find a solution and broadcast it to the network, ensuring that the system does not enter a state of indefinite waiting, and a leader is always elected, ensuring the continued operation of the blockchain network.

\subsubsection{Safety}

Safety is maintained in the Proof of Work Algorithm by ensuring that at most one leader is elected. The cryptographic puzzle's difficulty ensures that only the first process to find a valid nonce is elected as the leader. This mechanism prevents multiple processes from simultaneously declaring themselves as leaders, thus avoiding conflicts and ensuring consistent coordination.

\subsubsection{Complexity}

The complexity of the Proof of Work Algorithm can be evaluated in terms of time and communication overhead:

\begin{itemize}
    \item \textbf{Time Complexity}: the time complexity of the Proof of Work Algorithm is probabilistic and depends on the difficulty of the cryptographic puzzle. On average, it takes time proportional to the inverse of the target threshold. If the target is set such that \(T\) is the average time to solve the puzzle, then:

\[
T \approx \frac{1}{\text{Target}}
\]

This complexity arises because each process must repeatedly hash the block data with different nonces until a valid solution is found.
    \item \textbf{Communication Overhead}: the communication overhead of the Proof of Work Algorithm is low to moderate. The main communication occurs when a process broadcasts the solution to the network. Given that all nodes need to verify the solution, the number of messages exchanged is proportional to the number of processes, \(n\).
\end{itemize}

\subsubsection{Scalability}

The Proof of Work Algorithm scales effectively with the number of processes. The probabilistic nature of the algorithm ensures that, on average, a leader is elected within a predictable timeframe, regardless of the number of participating nodes. However, the computational resources required for solving the puzzle can increase significantly with higher network difficulty.

\subsubsection{Fault Tolerance}
Due to its decentralized nature the Proof of Work Algorithm is higly robust. The network can tolerate node failures as the remaining nodes continue competing to solve the puzzle. However, it can be slow to converge to a new leader if the network is partitioned or if many nodes fail simultaneously.
%\subsubsection{Energy Consumption}

%One of the significant drawbacks of the Proof of Work Algorithm is its high energy consumption. The continuous and intensive computation required to solve the cryptographic puzzle results in substantial energy usage, making it less environmentally friendly and cost-effective for large-scale deployment.

In summary, while the Proof of Work Algorithm provides a robust and secure method for leader election in blockchain networks, its high energy consumption and probabilistic time complexity are notable drawbacks. Its design ensures correctness, liveness, and safety, making it suitable for decentralized systems where security and consensus are critical.

%\subsubsection{Correctness}

%The correctness of the Proof of Work Algorithm can be shown by proving that it guarantees the election of a single leader who successfully solves the cryptographic puzzle.

%\subsubsection{Liveness}

%Every process eventually either solves the puzzle or acknowledges another process as the solver.

%\subsubsection{Safety}

%At most one leader is elected since the cryptographic puzzle ensures that only the first process to find a valid nonce is elected.

%\subsubsection{Complexity}

%The time complexity of the Proof of Work Algorithm is probabilistic and depends on the difficulty of the cryptographic puzzle. On average, it takes time proportional to the inverse of the target threshold.


\section{Proof of Stake}

% The Proof of Stake (PoS) algorithm is a widely-used method for leader election in blockchain-based distributed systems. It is particularly suitable for decentralized networks where security, energy efficiency, and consensus are critical. The algorithm proceeds as follows:
The Proof of Stake (PoS) algorithm is a widely-used method for leader election in blockchain-based distributed systems. This algorithm is effective in scenarios where nodes are selected based on their stake (i.e., the amount of cryptocurrency they hold) rather than their computational power. It operates under the assumption that each process has a unique stake, and the process with the highest stake has a higher probability of being elected as the leader. The algorithm is favored in environments where energy efficiency, security, and consensus are critical, making it suitable for blockchain networks. Its probabilistic approach ensures that a leader is always elected, thus providing robust coordination \cite{Ref12}.

\subsection{System Model}
Consider a distributed system consisting of \(n\) processes \(P_1, P_2, \ldots, P_n\) that are part of a blockchain network. Each process has a stake in the form of cryptocurrency, and the process with the highest stake has a higher probability of being elected as the leader:

\[
\text{Probability}(P_i) = \frac{\text{Stake}(P_i)}{\sum_{j=1}^{n} \text{Stake}(P_j)}
\]
The Proof of Stake Algorithm can be described in the following steps:

\subsubsection{Stake Initialization}

Each process \(P_i\) announces its stake to the network. The probability of being elected as the leader is proportional to the stake held by each process.

\begin{equation}
P_i \rightarrow \forall P_j: \text{Stake}(P_i)
\end{equation}

\subsubsection{Leader Selection}

A process is selected as the leader based on the probability distribution of the stakes. The process with the highest stake has a higher chance of being selected.

\begin{equation}
\text{Select } P_i \text{ with probability } \frac{\text{Stake}(P_i)}{\sum_{j=1}^{n} \text{Stake}(P_j)}
\end{equation}

\subsubsection{Validation and Consensus}

Once a leader is selected, it proposes a new block. Other processes validate the proposed block. If the block is valid, it is added to the blockchain.

\begin{equation}
P_i \rightarrow \forall P_j: \text{New Block}(P_i)
\end{equation}

\subsubsection{Algorithm Termination}

The algorithm terminates when a single leader is elected and a new block is added to the blockchain. Each process \(P_i\) updates its blockchain to include the new block:

\begin{equation}
\text{Blockchain}_i = \text{Blockchain}_i + \text{New Block}(P_i)
\end{equation}

\subsection{Pseudocode}

The following pseudocode provides a clear and concise representation of the Proof of Stake Algorithm. The pseudocode is structured into several key parts, reflecting the steps each process follows to ensure a leader is elected within a blockchain network.

\begin{itemize}
    \item \textbf{Stake Initialization:} Each process announces its stake to the network.
    \item \textbf{Leader Selection:} A process is selected as the leader based on the probability distribution of the stakes.
    \item \textbf{Validation and Consensus:} The selected leader proposes a new block, and other processes validate and add the block to the blockchain.
\end{itemize}
The pseudocode for the Proof of Stake Algorithm is as follows:

\begin{verbatim}
def announce_stake(P):
    broadcast_stake(P.id, P.stake)

def select_leader():
    total_stake = sum(P.stake for P in network)
    random_value = random() * total_stake
    cumulative_stake = 0
    for P in network:
        cumulative_stake += P.stake
        if cumulative_stake >= random_value:
            return P

def propose_block(leader):
    new_block = create_block(leader)
    broadcast_block(new_block)

def validate_block(new_block):
    if verify_block(new_block):
        update_blockchain(new_block)
\end{verbatim}

\subsubsection{Explanation of Pseudocode}

\begin{itemize}
    \item \texttt{announce\_stake(P)}: This function is called by each process to announce its stake to the network.
    \item \texttt{select\_leader()}: This function selects a leader based on the probability distribution of the stakes.
    \item \texttt{propose\_block(leader)}: This function is called by the selected leader to propose a new block.
    \item \texttt{validate\_block(new\_block)}: This function is called by other processes to validate the proposed block and update the blockchain.
\end{itemize}
This pseudocode provides a high-level view of the Proof of Stake Algorithm's operations, highlighting the interactions between processes and the steps involved in electing a new leader and adding a new block to the blockchain.

\subsection{Analysis}
The analysis of the Proof of Stake (PoS) Algorithm focuses on several key metrics: correctness, liveness, safety, and complexity. These metrics are critical in evaluating the performance and reliability of the algorithm in blockchain-based distributed systems.

\subsubsection{Correctness}

The correctness of the Proof of Stake Algorithm is guaranteed by its design. The algorithm ensures that a single leader is elected based on the stake distribution. This is achieved by selecting a process proportionally to its stake, ensuring that the higher the stake, the higher the probability of being elected as the leader. Consequently, only one process is chosen, maintaining the integrity of the election process.

\subsubsection{Liveness}

Liveness is a key property that ensures that the election process eventually completes, and a leader is elected. In the Proof of Stake Algorithm, every process either becomes the leader if it is selected based on its stake or acknowledges another process as the leader. This guarantees that the system does not enter a state of indefinite waiting, and a leader is always elected, ensuring the continued operation of the blockchain network.

\subsubsection{Safety}

Safety is maintained in the Proof of Stake Algorithm by ensuring that at most one leader is elected. The selection process based on stake distribution guarantees that only one process is chosen, preventing multiple processes from simultaneously declaring themselves as leaders. This mechanism ensures that there is always a single, unique leader in the system, thus avoiding conflicts and ensuring consistent coordination.

\subsubsection{Complexity}

The complexity of the Proof of Stake Algorithm can be evaluated in terms of time and communication overhead:
\begin{itemize}
    \item \textbf{Time Complexity}: the time complexity of the Proof of Stake Algorithm is \(O(1)\) for leader selection, as it involves generating a random value and traversing the list of processes once. This constant time complexity makes the PoS algorithm highly efficient for leader selection.
    \item \textbf{Communication Overhead}: the communication overhead of the Proof of Stake Algorithm is low. The main communication occurs during the announcement of stakes and the broadcast of the new block. Given \(N\) processes, the message complexity in the worst case can be described as:
\[
\Theta(N) = N
\]
This complexity arises because each process announces its stake and participates in the validation of the new block.
\end{itemize}

\subsubsection{Scalability}

The Proof of Stake Algorithm scales effectively with the number of processes. Its low communication overhead and constant time complexity for leader selection ensure that it can handle large-scale blockchain networks efficiently. This makes PoS a suitable choice for modern blockchain systems requiring high scalability.

\subsubsection{Fault Tolerance}
The Proof of stake algorithm is robust due to its low communication overhead and energy efficiency. The network can handle node failures as the leader selection is based on stake, and nodes with higher stakes are more incentivized to maintain the network's integrity. However, it relies on the assumption that the distribution of stakes does not lead to centralization, which can be a point of failure if not managed properly.

%\subsubsection{Energy Efficiency}

%One of the significant advantages of the Proof of Stake Algorithm over Proof of Work (PoW) is its energy efficiency. PoS eliminates the need for intensive computational work, significantly reducing energy consumption and making it a more environmentally friendly and cost-effective solution for blockchain networks.

In summary, the Proof of Stake Algorithm provides an energy-efficient, scalable, and robust method for leader election in blockchain networks. Its design ensures correctness, liveness, and safety, making it an attractive alternative to PoW for modern blockchain applications.

%\subsubsection{Correctness}

%The correctness of the Proof of Stake Algorithm can be shown by proving that it guarantees the election of a single leader based on the stake distribution.

%\subsubsection{Liveness}

%Every process eventually either becomes the leader or acknowledges another process as the leader.

%\subsubsection{Safety}

%At most one leader is elected since the selection process ensures that only one process is chosen based on the stake distribution.

%\subsubsection{Complexity}

%The time complexity of the Proof of Stake Algorithm is \(O(1)\) for leader selection, as it involves generating a random value and traversing the list of processes once.


%\section{Bully}

%The Bully algorithm is a well-known method for leader election in distributed systems. It is particularly effective in scenarios where nodes can directly communicate with each other. The algorithm proceeds as follows:
%The Bully algorithm is a well-known method for leader election in distributed systems to dynamically select a coordinator from among a group of processes.  It is particularly effective in scenarios where nodes can directly communicate with each other and it operates under the assumption that each process has a unique identifier (ID), and the process with the highest ID among the non-faulty processes is elected as the coordinator. It is favored in environments where direct communication between all nodes is possible, making it suitable for smaller, tightly-coupled distributed systems. Its straightforward approach and deterministic nature ensure that a leader is always elected if the system remains functional, thus providing robust coordination.

%\subsection{Assumptions}

%The Bully algorithm makes several key assumptions:
%\begin{enumerate}
%    \item The system is synchronous.
%    \item Processes may fail at any time, including during the execution of the algorithm.
%    \item A failed process stops working and resumes upon recovery.
%    \item A failure detector is present to identify failed processes.
%    \item Message delivery between processes is reliable.
%    \item Each process knows its own ID and address, as well as those of all other processes.
%\end{enumerate}

%\subsection{Algorithm Steps}

%\begin{enumerate}
%    \item \textbf{Election Initiation}: When a process \( P \) detects the failure of the current coordinator or when it recovers from a failure, it starts the election process if it is not the highest ID process. \( P \) sends an Election message to all processes with higher IDs.
%    \item \textbf{Handling Responses}: If \( P \) does not receive any Answer messages from higher ID processes within a timeout period, it declares itself the coordinator by sending a Victory message to all other processes. If \( P \) receives an Answer message, it waits for a Victory message from the higher ID process that responded.
%    \item \textbf{Receiving Election Messages}: When a process receives an Election message from a lower ID process, it responds with an Answer message. It then starts its own election if it is not already participating in one by sending Election messages to higher ID processes.
%    \item \textbf{Receiving Victory Messages}: When a process receives a Victory message, it updates its coordinator to the sender of the Victory message.
%\end{enumerate}


%\subsection{Analysis}

%\begin{itemize}
%    \item \textbf{Safety}: The algorithm guarantees that at most one coordinator is elected. If two processes attempt to become the coordinator simultaneously, the one with the higher ID will prevail, as lower ID processes will concede upon receiving Answer messages from higher ID processes.
%    \item \textbf{Liveness}: The algorithm ensures that eventually a coordinator will be elected, provided the system is synchronous and the failure detector works correctly. Even if multiple failures occur, the election process will continue until a coordinator is selected.
%    \item \textbf{Message Complexity}: The worst-case scenario for message exchange occurs when the process with the lowest ID initiates an election. This requires \( O(N^2) \) messages, where \( N \) is the number of processes. This includes Election messages, Answer messages, and Victory messages.
%\end{itemize}

%\subsection{Mathematical Formulation}

%Given \( N \) processes, the message complexity in the worst case can be described as:
%\[ \Theta(N^2) \]
%where:
%\[ (N-1) + (N-2) + \cdots + 1 = \frac{N(N-1)}{2} \approx \Theta(N^2) \]
%This complexity arises because each process with a lower ID sends Election messages to all higher ID processes, and each higher ID process responds with an Answer message if it is active.